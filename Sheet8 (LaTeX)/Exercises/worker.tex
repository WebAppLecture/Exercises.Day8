\Exercise{Was mit WebWorker machen}
%
\par Ihre Aufgabe ist es die Macht der Mehrkernprozessoren auszunutzen. Dies ist über WebWorker möglich. Der WebWorker soll in dieser Aufgabe die Mandelbrotmenge berechnen (Mehr Informationen sind unter \url{http://de.wikipedia.org/wiki/Mandelbrot-Menge} verfügbar). Der Ablauf sieht dabei folgendermaßen aus:
%
\begin{itemize}
\item Canvas, einen Button und $4$ Number Boxen ($x_s$, $x_e$, $y_s$, $y_e$) erstellen
\item Den WebWorker Code erstellen – wird durch Klick auf den Button mit den Argumenten Offset-X, Offset-Y sowie Höhe und Breite des Canvas gestartet
\item Der WebWorker Code gibt ein Array zurück in dem alle zu zeichnenden Pixel inkl. Farben stehen
\item Der Code im WebWorker besitzt folgende Funktionalität (über alle Pixel des Canvas durchgehen):
%
	\begin{enumerate}
	\item $x_0$ (Pixel, skaliert zwischen $x_s$ und $x_e$) und $y_0$ (Pixel, skaliert zwischen $y_s$ und $y_e$) setzen
	\item $x_0$ sollte z.B. zwischen $-2.5$ und $1$ liegen und $y_0$ zwischen $-1$ und $1$
	\item $x$ und $y$ auf $0$ setzen
	\item Solange $x^2 + y^2 < 4$ und noch keine $\sim 1000$ Iterationen durchlaufen wurden:
%
		\begin{enumerate}
		\item $x_t$ auf $x^2 - y^2 + x_0$ setzen
		\item $y$ auf $2xy + y_0$ setzen und anschließend $x$ auf $x_t$ setzen
		\end{enumerate}
%
	\item Am Ende den Pixel in das Rückgabearray aufnehmen wenn $x^2 + y^2 < 4$ gilt
	\end{enumerate}
%
\end{itemize}
%
\par Bitte beachten: Ein Klick auf den Button soll die vorherige Berechnung (falls vorhanden) zunächst abbrechen!