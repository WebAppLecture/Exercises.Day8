\Exercise{Ein eigenes jQuery Plugin schreiben}
%
\par Es soll ein Plugin für ein sog. Sliding Panel geschrieben werden. Ein
Sliding Panel (manchmal auch horizontales Accordion genannt) ist eine
Möglichkeit Informationen zu verpacken um diese bei Bedarf anzuzeigen, z.B.:
%
\begin{figure}[!h]
\centering
\includegraphics{Exercises/Figures/pluginclose.png}
\end{figure}
%
\par Nach dem Öffnen eines Panels soll sich der Ausschnitt der Seite
folgendermaßen darstellen:
%
\begin{figure}[!h]
\centering
\includegraphics{Exercises/Figures/pluginopen.png}
\end{figure}
%
\par Als Grundlage für das Plugin soll folgende HTML-Seite verwendet werden:
\url{http://html5.florian-rappl.de/Aufgabe38}.html. In dieser Datei sind
bereits zwei solcher Panels enthalten, zusammengefasst in einer CSS-Klasse
\jvar{panels}.
%
\par Schreiben Sie nun ein jQuery Plugin, welches über eine externe Datei
(\emph{jquery.panels.js}) eingebunden wird. Anschließend soll das Plugin über
ein internes Skript auf die beiden Panels angewandt werden. Das Plugin soll
folgende Eigenschaften besitzen:
%
\par Beim Hovern mit der Maus soll der aktuelle Eintrag vergrößert und alle
anderen verkleinert werden. Hierbei soll der aktuelle Eintrag auf \qty{360}{px}
mit einer größeren Schriftgröße (z.B. \qty{150}{px}) verbreitert werden. Alle
anderen Einträge müssen demnach kleiner werden (auch kleinere Schriftgrößen
verwenden, z.B. \qty{135}{px} und \qty{50}{px}!). Ist die Maus über keinem
Element so soll die Ausgangsposition wieder eingenommen werden (z.B.
\qty{180}{px} und \qty{100}{px}). Alle Veränderungen sollen durch Animationen
mit einer Animationszeit von \qty{200}{ms} angezeigt werden.
%
\par Nachdem das Plugin in dieser Fassung funktioniert sollen dynamische
Optionen durch z.B. verstellbare Größen ermöglicht werden. Verwenden Sie
hierbei die \jfunc{extend} Methode von jQuery.